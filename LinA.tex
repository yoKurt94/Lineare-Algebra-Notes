\documentclass{article}
\usepackage{amsmath}
\usepackage{amssymb}
\usepackage{tabularx}


\title{Lineare Algebra}
\author{Yannik Hörnschemeyer}
\date{\today}

\begin{document}

\maketitle

\date{Mittwoch, 18.10.23}

\section{Aussagenlogik}
\subsubsection*{Definition 1.1} Eine Aussage ist ein Satz, der entweder wahr oder falsch ist.\\
Beispiele:
\begin{itemize}
    \item "8 ist eine gerae Zahl." (wahre Aussage)
    \item "4 ist eine Primzahl." (falsche Aussage)
    \item "Es gibt unendlich viele Primzahlzwillinge." (bei dieser Aussge ist der Wahrheitsgehalt unbekannt. Nur weil wir den Wahrheitsgehalt noch nicht kennen heißt das nicht, dass es keine Aussge ist.)
    \item "Heute ist ein schöner Tag." (keine Aussage, da der Wahrheitsgehalt von der Person abhängt, die die Aussage macht.)
\end{itemize}

Aus schon gegebenen Aussagen können wir neue Aussagen bilden.\\
\subsubsection*{Definition 1.2}
Es seien A und B Aussagen. 

\begin{center}
    \begin{tabular}{|c|c|c|c|c|c|c|}
        \hline
        A & B & ¬A & A $\wedge$ B & A $\vee$ B & A $\rightarrow$ B & A $\leftrightarrow$ B \\
        \hline
        \hline
        w & w & f & w & w & w & w \\
        w & f & f & f & w & f & f \\
        f & w & w & f & w & w & f \\
        f & f & w & f & f & w & w \\
        \hline
    \end{tabular}
\end{center}

\subsubsection*{Bemerkung}
\begin{enumerate}
    \item $\neg A$ wird gesprochen 'nicht A'. 
    \item $A \wedge B$ wird gesprochen 'A und B'. 
    \item $A \vee B$ wird gesprochen 'A oder B'. 
    \item $A \Rightarrow B$ wird gesprochen 'A impliziert B', 'Aus A folgt B', 'A ist hinreichend für B', 'B ist notwendig für A', 'Wenn A dann B'. 
    \item $A \Leftrightarrow B$ wird gesprochen 'A äquivalent B', 'A ist notwendig und hinreichend für B', 'A genau dann wenn B'
\end{enumerate} 
    
\subsubsection*{Bemerkung}
Warum folgt aus einer falschen Aussage etwas Wahres? \\
In Beweisen müssen wir zeigen, dass etwas immer wahr ist. Wenn zum Beispiel $n$ gerade ist, dann $n^2$ gerade. Wenn $n$ ungerade, dann müssten wir diesen Fall im Beweis auch abdecken. Durch die Definition der Implikation können wir diesen Fall aber ignorieren, da die Aussage dann automatisch wahr ist. \\

\subsubsection*{Lemma 1.3}
Sei A eine Aussage. Dann ist $A \vee \neg A$ wahr.\\

\subsubsection*{Beweis}
Wir untersuchen die zwei Fälle für A: A ist wahr oder A ist falsch. \\
Wir betrachten die Wahrheitstabelle von $A \vee \neg A$ \\
\begin{center}
    \begin{tabular}{|c|c|c|}
        \hline
        A & $\neg$ A & A $\vee$ $\neg$ A \\
        \hline
        \hline
        w & f & w \\
        f & w & w \\
        \hline
    \end{tabular}
\end{center} $\square$ \\


Hinweis: Ein Beweis per Wahrheitstafel ist eine valide Beweismethode. \\
Hinweis: Eine Tautologie ist eine Aussage, die immer wahr ist. \\

\subsubsection*{Bemerkung}
Das $\neg$ Zeichen bindet stärker als die anderen Verknüpngen. Beispiel: \\
$\neg A \vee B$ ist äquivalent zu $(\neg A) \vee B$ \\
Außerdem gibt es die Konvention, dass das 'und' und das 'oder' stärker bindet als die Implikation. \\
Die Reihenfolge der Stärke der Bindung ist also: $\neg, \wedge, \vee, \rightarrow$ \\

%folgede Tabelle kontrollieren

\subsubsection*{Lemma 1.4}
Es seien $A$, $B$ und $C$ Aussagen. Dann sind die fogenden Aussagen jeweils äquivalent: \\
\begin{enumerate}
    \item $A \rightarrow B$ und $\neg A \vee B$
    \item $A \leftrightarrow B$ und $(A \rightarrow B) \wedge (B \rightarrow A)$
    \item $A$ und $\neg \neg A$
    \item $A$ und $\neg A \rightarrow $ falsch
    \item $A \rightarrow B$ und $\neg B \rightarrow \neg A$
    \item $A \wedge B$ und $B \wedge A$
    \item $A \vee B$ und $B \vee A$
    \item $(A \wedge B) \vee C$ und $A \wedge (B \vee C)$
    \item $(A \vee B) \vee C$ und $A \vee (B \vee C)$
    \item $A \wedge (B \vee C)$ und $(A \wedge B) \vee (A \wedge C)$
    \item $A \vee (B \wedge C)$ und $(A \vee B) \wedge (A \vee C)$
    \item $\neg (A \wedge B)$ und $\neg A \vee \neg B$
    \item $\neg (A \vee B)$ und $\neg A \wedge \neg B$
\end{enumerate}

\subsubsection*{Bemerkungen}
Wenn die Aussagen äquivalent sind? Die linke Aussage ist äquivalent $\leftrightarrow$ zur rechten Aussage und damit immer wahr. \\
zu 1: Man kann in a und b auch statt $\rightarrow$ und $\leftrightarrow$ auch $\vee$ und $\wedge$ nutzen. \\
zu 4: Aufbau eines textit{Widerspruchsbeweis}. d rechtfertigt also den Widerspruchsbwesei. \\
zu 5: \textit{Kontraposition} von a. \\
zu 6 und 7: \textit{Kommutativität} von $\wedge$. \\
zu 8 und 9: \textit{Assoziativität} von $\wedge$ und $\vee$. Wenn ich mehrere Aussagen mit $\wedge$ oder $\vee$ verknüpfe, dann ist es egal in welcher Reihenfolge man die Klammern setzt (und ob man sie setzt). \\
zu 10 und 11: \textit{Distributivität} von $\wedge$ und $\vee$. \\
zu 12 und 13: \textit{De Morgan'sche Regel} (oder 'Gesetze'). \\

\subsubsection*{Beweis (Aussage 1)}
Beweis per Wahrheitstafel. \\
\begin{center}
    \begin{tabular}{|c|c|c|c|c|}
        \hline
        A & B & $\neg$ A & A $\rightarrow$ B & $\neg$ A $\vee$ B \\
        \hline
        \hline
        w & w & f & w & w \\
        w & f & f & f & f \\
        f & w & w & w & w \\
        f & f & w & w & w \\
        \hline
    \end{tabular}
\end{center}
Wenn wir die letzten beiden Spalten vergleichen, sehen wir, dass die Aussagen äquivalent sind. \\
Damit ist die Aussage bewiesen. $\square$\\

% \begin{thebibliography}{9}
%     \bibitem{book:howToProveIt}
%     Velleman, Daniel J.,
%     \emph{How To Prove It: A Structured Approach},
%     Camebridge University Press, 2006.
%     \end{thebibliography}


\end{document}
