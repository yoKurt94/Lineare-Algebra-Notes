\documentclass{article}
\usepackage{amsmath}
\usepackage{amssymb}
\usepackage{tabularx}


\title{Lineare Algebra}
\author{Yannik Hörnschemeyer}
\date{\today}

\begin{document}

\maketitle

\section{Aussagenlogik}
Definition 1.1: Eine Aussage ist ein Satz, der entweder wahr oder falsch ist.\\
Beispiele:
\begin{itemize}
    \item "8 ist eine gerae Zahl." (wahre Aussage)
    \item "4 ist eine Primzahl." (falsche Aussage)
    \item "Es gibt unendlich viele Primzahlzwillinge." (bei dieser Aussge ist der Wahrheitsgehalt unbekannt. Nur weil wir den Wahrheitsgehalt noch nciht kennen heißt es nicht, dass das hier keine Aussge ist.)
    \item "Heute ist ein schöner Tag." (keine Aussage, da der Wahrheitsgehalt von der Person abhängt, die die Aussage macht.)
\end{itemize}

Aus schon gegebenen Aussagen können wir neue Aussagen bilden.\\
Definition 1.2: Es seien A und B Aussagen. 

% Ergänze 'nciht a' und "nicht b" a und B a oder b, a impliziert b, a äquivalent b 
\begin{center}
    \begin{tabular}{|c|c|c|c|c|c|c|}
        \hline
        A & B & ¬A & A $\wedge$ B & A $\vee$ B & A $\rightarrow$ B & A $\leftrightarrow$ B \\
        \hline
        \hline
        w & w & f & w & w & w & w \\
        w & f & f & f & w & f & f \\
        f & w & w & f & w & w & f \\
        f & f & w & f & f & w & w \\
        \hline
    \end{tabular}
\end{center}

\subsubsection*{Bemerkung:}
$\neg A$ wird gesprochen 'nicht A'. \\
$A \wedge B$ wird gesprochen 'A und B'. \\
$A \vee B$ wird gesprochen 'A oder B'. \\
$A \rightarrow B$ wird gesprochen 'A impliziert B'. \\
$A \leftrightarrow B$ wird gesprochen 'A äquivalent B'. \\
Synonyme für $A \Rightarrow B$: Aus A folgt B, A ist hinreichend für B, B ist notwendig für A, Wenn A dann B \\
Synonyme für $A \Leftrightarrow B$: A ist äquivalent zu B, A ist notwendig und hinreichend für B, A genau dann wenn B \\
\subsubsection*{Bemerkung}
Warum folgt aus einer Falschen Aussage etwas wahres? \\
In Beweisen müssen wir zeigen, dass etwas immer wahr ist. Beispiel wenn n gerade, dann n hoch 2 gerade. Wenn n ungerade, dann müssten wir diesen Fall im Beweis auch abdecken. Durch die Definition der Implikation können wir diesen Fall aber ignorieren, da die Aussage dann automatisch wahr ist. \\

\subsubsection*{Lemma 1.3:}
Sei A eine Aussage. Dann ist $A \vee \neg A$ wahr.\\

\subsubsection*{Beweis:}
Wie untersuchen die zwei Fälle für A: A ist wahr oder A ist falsch. \\
Wir betrachten die Wahrheitstabelle von $A \vee \neg A$ \\
\begin{center}
    \begin{tabular}{|c|c|c|}
        \hline
        A & $\neg$ A & A $\vee$ $\neg$ A \\
        \hline
        \hline
        w & f & w \\
        f & w & w \\
        \hline
    \end{tabular}
\end{center} $\square$ \\

Hinweis: Ein Beweis per Wahrheitstafel ist eine valide Beweismethode. \\
Eine Tautologie ist eine Aussage, die immer wahr ist. \\

\subsubsection*{Bemerkung:}
Das $\neg$ Zeichen bindet stärker als die anderen Verknüpngen. Beispiel: \\
$\neg A \vee B$ ist äquivalent zu $(\neg A) \vee B$ \\
Außerdem gibt es die Konvention dass das 'und' und das 'oder' stärker bindet als die Implikation. \\
Die Reichenfolge der Stärke der Bindung ist also: $\neg, \wedge, \vee, \rightarrow$ \\

%folgede Tabelle kontrollieren

\subsubsection*{Lemma 1.4:}
Es seien $A$, $B$ und $C$ Aussagen. Dann sind die fogenden Aussagen jeweils äquivalent: \\
\begin{enumerate}
    \item a $A \rightarrow B$ und $\neg A \vee B$
    \item b $A \leftrightarrow B$ und $(A \rightarrow B) \wedge (B \rightarrow A)$
    \item c $A$ und $\neg \neg A$
    \item d $A$ und $\neg A \rightarrow $ falsch
    \item e $A \rightarrow B$ und $\neg B \rightarrow \neg A$
    \item f $A \wedge B$ und $B \wedge A$
    \item g $A \vee B$ und $B \vee A$
    \item h $(A \wedge B) \vee C$ und $A \wedge (B \vee C)$
    \item i $(A \vee B) \vee C$ und $A \vee (B \vee C)$
    \item j $A \wedge (B \vee C)$ und $(A \wedge B) \vee (A \wedge C)$
    \item k $A \vee (B \wedge C)$ und $(A \vee B) \wedge (A \vee C)$
    \item l $\neg (A \wedge B)$ und $\neg A \vee \neg B$
    \item m $\neg (A \vee B)$ und $\neg A \wedge \neg B$
\end{enumerate}

\subsubsection*{Bemerkungen:}
Wenn die Aussagen äquivalent sind? Die linke Aussage ist äquivalent $\leftrightarrow$ zur rechten Aussage und damit immer wahr. 
zu a: Man kann in a und b auch statt $\rightarrow$ und $\leftrightarrow$ auch $\vee$ und $\wedge$ nutzen. \\
zu d: d zeigt den Aufbau eines textit{Widerspruchsbeweis}. d rechtfertigt also den Widerspruchsbwesei. \\
zu e: e ist die \textit{Kontraposition} von a. \\
zu f und g: f und g ist die \textit{Kommutativität} von $\wedge$. \\
zu h und i: h und i ist die \textit{Assoziativität} von $\wedge$ und $\vee$. Wenn ich mehrere Aussagen mit $\wedge$ oder $\vee$ verknüpfe, dann ist es egal in welcher Reihenfolge man die Klammern setzt (und ob man sie setzt). \\
zu j und k: j und k ist die \textit{Distributivität} von $\wedge$ und $\vee$. \\
zu l und m: l und m ist die \textit{De Morgan'sche Regel} (oder 'Gesetze'). \\

\subsubsection*{Beweis (Aussage a):}
Beweis per Wahrheitstafel. \\
\begin{center}
    \begin{tabular}{|c|c|c|c|c|}
        \hline
        A & B & $\neg$ A & A $\rightarrow$ B & $\neg$ A $\vee$ B \\
        \hline
        \hline
        w & w & f & w & w \\
        w & f & f & f & f \\
        f & w & w & w & w \\
        f & f & w & w & w \\
        \hline
    \end{tabular}
\end{center}
Wenn wir die letzten beiden Spalten vergleichen sehen wir, dass die Aussagen äquivalent sind. \\
Damit ist die Aussage bewiesen. $\square$\\
% Beweissymbol einfügen

\begin{thebibliography}{9}
    \bibitem{book:howToProveIt}
    Velleman, Daniel J.,
    \emph{How To Prove It: A Structured Approach},
    Camebridge University Press, 2006.
    \end{thebibliography}


\end{document}
